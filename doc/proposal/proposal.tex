\documentclass{article}
\usepackage{color}
\usepackage{graphicx}
\usepackage[caption=false]{subfig}
\usepackage{amsmath}
\allowdisplaybreaks
% \usepackage{amsthm}
\usepackage{amsfonts}
\usepackage{mathtools}
\usepackage{siunitx}
\usepackage[bookmarks]{hyperref}
\usepackage[capitalize]{cleveref}
\usepackage[utf8]{inputenc}
\usepackage{algpseudocode}
\usepackage{algorithm}

\graphicspath{{figures/}}

% Environments
% \newtheorem{theorem}{Theorem}
% \newtheorem{definition}{Definition}
% \newtheorem{problem}{Problem}
\newtheorem{objective}{Objective}
\newtheorem{example}{Example}

\crefname{problem}{Problem}{Problems}
\crefname{section}{Sec.}{Secs.}
\crefname{theorem}{Thm.}{Thms.}
\crefname{definition}{Def.}{Defs.}
\crefname{objective}{Obj.}{Obj.}

% Aliases

\newcommand*{\xd}{\dot{x}}
\newcommand*{\R}{\mathbb{R}}
\newcommand*{\N}{\mathbb{N}}

%% Temporal logic symbols
% \newcommand{\not}{\neg}
% \newcommand{\and}{\wedge}
% \newcommand{\or}{\vee}
\newcommand{\Next}{\mathbf{X}}
\newcommand{\Always}{\mathbf{G}}
\newcommand{\Event}{\mathbf{F}}
\newcommand{\luntil}{\mathbf{U}}
\newcommand{\Implies}{\Rightarrow}
\newcommand{\Not}{\lnot}
\newcommand{\True}{\top}

% \newcommand*{\iff}{\Leftrightarrow}

\newcommand*{\psat}[1]{[[#1]]}

\newcommand*{\fran}[1]{\textcolor{blue}{#1}}

\begin{document}

\section{Introduction}
\label{sec:introduction}

In the field of formal methods, a wide variety of system models are analyzed 
against a formal specification, commonly represented as a temporal logic
formula, with the goal of verifying whether the specification is satisfied, or
synthesizing controls that guarantee that such is the case. In the literature,
several tools and algorithms have been proposed for systems modeled as
finte-state transition systems, linear and non-linear Ordinary Differential
Equations (ODEs) and hybrid systems.  However, the analysis of systems governed
by Partial Differential Equations (PDEs) has not been attempted yet. Our goal is
to develop tools and algorithms that address the common problems in formal
methods for these kinds of systems.

Consider, for example, a steel plate whose temperature at each point $x$ and
time $t$ is given by $u(x, t)$. The evolution of the temperature follows the
well-known heat equation, which is a PDE with a spatial domain defined by the
steel plate. We want to design a control strategy
to apply heat to the boundary of the plate in such a way that the temperature in
the left half of the plate oscillates forever between being higher than \SI{300}{\kelvin} 
and lower than \SI{200}{\kelvin} at all points, while the right half oscillates forever in the
opposite way. In addition to the usual temporal operators such as ``forever", we
also need to consider spatial operators, such as ``at all points in the left
half of the plate".

\section{Objectives}
\label{sec:objectives}

Our main objective for this project is to develop a formal methods framework for
the verification and control of systems governed by PDEs. Our overall approach
will consist in reducing the PDE to an ODE by applying the Finite Element Method
(FEM), and adapting existing tools and algorithms in formal methods to be able
to solve the resulting high-dimensional problem. This approach will constrain us
to PDEs that the FEM method can be applied to, which is not an overly restrictive
assumption, as all PDEs of engineering interest (Maxwell equations, Navier-Stokes equations,
momentum equation, diffusion and heat equations, etc) meet this criteria.
Additionally, in this project we will consider the resulting ODE to be linear,
which constrains the physical system to linear material behavior. We will
consider PDEs in 1D, 2D and 3D.

\begin{objective} \label{obj:spec}
    [Specification languages] We will develop formal specification languages
    suitable for describing properties of PDE systems. The languages will be
    able to capture both temporal and spatial features of the evolution of the
    system.
\end{objective}

Some problems will require specification languages with certain properties, such
as overall bounded or unbounded time, explicit time bounds for specific
properties, discrete or continuous time semantics, stochastic properties, and
different types of spatial features, such as extremes and averages over
a spatial domain or universal and existential properties. We will focus on
adding to existing temporal logics, such as LTL and STL, the necessary elements
to describe spatial properties. 

\begin{objective} \label{obj:abst_based}
    [Abstraction-based verification and synthesis] We will develop a
    verification and synthesis algorithm for PDE systems, with specifications
    given in the languages proposed above, based on the construction of
    abstractions and finite-state models.
\end{objective}

With \cref{obj:abst_based} we will extend the classical formal methods approach
for verification and control to PDE systems. Specifically, we will translate
our specifications to automata, build a finite-state abstraction of the PDE
system, and solve the verification or control problem on the product between
both. This objective presents two key challenges: first, the finite automaton
obtained from the specification needs to represent spatial properties over
spatial domains that are infinite sets; second, the state space of a PDE system
is infinite-dimensional, and classical abstraction-based
tools for ODE systems do not scale well in the number of dimensions. We will
address these challenges by first obtaining a finite-dimensional approximation
of the PDE system using the FEM, and then exploiting its special coupled
structure to work with lower-dimensional systems.

\begin{objective} \label{obj:opt_based}
    [Optimization-based verification and synthesis] We will develop a
    verification and synthesis algorithm for PDE systems based on optimization
    techniques and receding horizon control.
\end{objective}

For this objective, we will build upon the recent advances in optimal control
for linear systems with temporal logic constraints. We will develop a mixed
integer linear encoding for both the PDE system and the specification, which we can then solve
for a variety of cost functions, such as energy spent or robustness of the
specification. In this approach, both the system trajectory and the
specification need to be discretized in space and time so they can be encoded 
as MILP constraints.
The main challenge will be to apply the discretization steps in such a way that
optimality and satisfaction guarantees obtained from the MILP apply to the PDE
system and the given specification.

We will implement our algorithms as publicly available Python packages that can
be easily incorporated as part of other projects.

\section{Significance}
\label{sec:significance}

The work proposed here will advance the state of the art in two different areas:
the theory and application of formal methods and the analysis of PDEs.

Regarding the field of formal methods, this work will extend the work done in
the formulation and solution of
verification and synthesis problems that have been considered in the literature for a
large class of systems, including finite-state systems, linear and non-linear
ODE systems and hybrid systems. The proposed research will add a new class of
systems never considered before: those governed by PDEs. Our approach will also
require the adaptation of existing tools and algorithms used in the analysis of
ODE systems, which will improve their scalability with the number of dimensions
of the system.

On the side of the analysis of PDEs, this work will allow the automatic
verification and design of systems
with complex specifications stated in a high-level and user-friendly language.
This process is also guaranteed to be correct.

\section{Research Plan}
\label{sec:research_plan}

\subsection{Specification Language}
\label{sub:specification_language}

In order to satisfy \cref{obj:spec}, our specification language must be able to
describe spatial properties of the system. By spatial we mean properties about
the state of the system in some spatial domain, such as:

\begin{itemize}
    \item The left half of a plate has a temperature above \SI{100}{\kelvin}
    \item The maximum displacement of a truss at the mid third of its
        length is less than \SI{2}{\cm}
\end{itemize}

We propose these properties to be encoded as simple predicates to be used as
part of already defined temporal logics, such as LTL or STL. A possible
definition would be the following: consider
each predicate $\lambda$ to be defined as a tuple $(Q_\lambda,
X_\lambda, \mu_\lambda)$ and represented using the syntax 
$Q_\lambda x \in X_\lambda : u(x) - \mu_\lambda(x) > 0$, with:

\begin{itemize}
    \item $Q_\lambda \in \{\forall, \exists\}$ the spatial quantifier,
    \item $X_\lambda$ the spatial domain of the predicate, which is a subset of
        the domain of the PDE
    \item $\mu_\lambda : X_\lambda \to \R$ a continuous and differentiable function 
        representing the reference profile.
\end{itemize}

The satisfaction of a
predicate with respect to the continuous-time trajectory of a PDE system,
$u(x,t)$, at time $t \in \R$ is defined as $u[t] \models \lambda_i \iff 
u(x, t) - \mu_\lambda(x) > 0$ for all $x \in X_\lambda$ if $Q_\lambda = \forall$ 
or for some $x \in X_\lambda$ otherwise.

These predicates are of particular interest, since they resemble first order
logic formulas and can be immediately used as part of LTL or STL formulae. In the
case of STL, we can easily define quantitative semantics, with the robustness $r$
of a predicate $\lambda$ being $r(\lambda_\lambda,u, t) = 
\min_{x \in X_\lambda} u(x, t)$ if $Q_\lambda = \forall$ and
$r(\lambda_\lambda,u, t) = \max_{x \in X_\lambda} u(x, t)$ otherwise.

Note that a similar approach can be used for similarly structured properties.
Indeed, the predicate $\exists x \in X: u(x) - \mu(x) > 0$ is equivalent to
$\max_{x \in X} (u(x) - \mu(x)) > 0$, and, more generally, we can write the 
predicate in functional form as $F_{X, \mu}(u) > 0$. From the point of view of
defining a specification language, there is no 
restriction on the type of functionals one can use to define predicates.
However, the verification and control synthesis algorithms that we will design
will impose certain limits, particularly the ability to "discretize" the
functional.

\begin{example}
    \label{ex:stl}

    Consider the following formula in STL for a heat equation with a single predicate
    $\lambda = (Q = \forall, X = [8, 9], \mu(x) = -(28x - 192))$, which we 
    represent using the syntax:

    \begin{equation}
        \psi = \Always_{[1,10]} \forall x \in [8,9] : u(x) - (28x - 192) > 0
    \end{equation}

    The specification reads in natural language: ``Always between 1 and 10 time
    units, at all points in a rod between lengths 8 and 9, the temperature
    profile is above the profile $\mu(x) = 28x - 192$."

\end{example}

An alternative to the predicate based approach is to use a spatial logic that
would allow us to describe spatial features of the system in terms of patterns.
For example, the property "the temperature of the plate follows a checkerboard
pattern" can be easily expressed in SpaTeL \cite{iman's work}.

\subsection{Finite Element Method}
\label{sub:Finite Element Method}

We provide in this section a brief summary of the Finite Element Method (FEM).
We present the method applied
to a simplified heat equation, although the same technique can be applied to other
PDEs.

Let $\Omega = (0, L) \subset \R$ be an open interval representing the interior
of a one-dimensional rod of length $L$; $\rho, c, \kappa > 0$ 
constants denoting density, capacity and conductivity of the rod's material respectively;
$g_0, g_L \in \R$ the boundary conditions at each end of the rod; and $u_0 :
\Omega \rightarrow \R$ an initial value for the temperature on the rod. 
The evolution of the temperature at
each point in the rod can be described by a function $u : \bar \Omega \times [0,
T] \rightarrow \R$, where $T > 0$ denotes the final time and can be infinity and
$\bar \Omega$ is the spatial domain, such that the following initial boundary
value problem (IBVP) is satisfied:

\begin{equation}\label{eq:pde}
    \Sigma^{H}(u_0, g) : \quad \left \{
    \begin{aligned}
        \rho c \frac{\partial u}{\partial t} - \kappa \frac{\partial^2
        u}{\partial^2 x} &= 0, \text{on } \Omega \times (0, T) \\
        u(0, t) &= g_0, \forall t \in (0, T) \\
        u(L, t) &= g_L, \forall t \in (0, T) \\
        u(x, 0) &= u_0(x), \forall x \in \Omega
    \end{aligned}
    \right.
\end{equation}

An important aspect of the FEM is the creation of a weak formulation of the
governing PDEs in \cref{eq:pde}. The purpose in deriving a weak form is to
reduce the second order spatial derivative of $u$ in \cref{eq:pde} to first
derivatives so that low order (in this case, linear) polynomials can be used to
approximate the field value $u$. We can construct a weak formulation of \cref{eq:pde} with the same set of
solutions in the following way: let $D$ be the set of sufficiently smooth 
real-valued functions on $\bar{\Omega} \times (0, T)$ such that all 
$u \in D$ satisfies $u(0, t) = g_0, u(L, t) = g_L, \forall t \in (0, T)$, 
and $V$ a similar set of time
independent functions such that all $w \in V$ satisfies $w(0) = w(L) = 0$. 
Find $u \in D$ such that for all $w \in V$,

\begin{equation}\label{eq:weak_pde}
    \begin{aligned}
        &\int_{\Omega} \frac{\partial w}{\partial x} \kappa \frac{\partial
        u}{\partial x} d \Omega + 
        \int_{\Omega} w \rho c \frac{\partial u}{\partial t} d \Omega = 0 \\
        &\int_{\Omega} w \rho c u(\cdot, 0) d \Omega =
        \int_{\Omega} w \rho c u^0 d \Omega \\
    \end{aligned}
\end{equation}

We now obtain an approximate solution to the weak formulation by considering
\cref{eq:weak_pde} with $u$ and $w$ in subspaces of $D$ and $V$.

Let $\{x_i\}_{i = 0}^{n +
1}$, where $x_0 = 0, x_{n+1} = L, x_i \in \Omega, i = 1,...,n$, be a partition of
$\bar\Omega$. Let $d_i(t), i = 0,...,n+1$ represent the
temperature of the rod at node $x_i$, with $d_0(t) = g_0, d_{n+1} = g_L$ and let $d = (d_1, ..., d_n)' \in
\R^n$. We define the following linear node shape function matrices for $i =
0,...,n+1$:

\begin{equation}
    N_i(x) = \begin{cases}
        \frac{x - x_{i - 1}}{x_i - x_{i - 1}} & i > 0, x \in [x_{i-1}, x_i] \\
        \frac{x_{i+1} - x}{x_{i+1} - x_{i}} & i < n+1, x \in [x_{i}, x_{i+1}] 
    \end{cases} 
\end{equation}

which results in the following linear interpolation of the field variable $u$ in
terms of its nodal values $d$:

\begin{equation}
    u^d(x, t) = \sum_{i=0}^{n+1} N_i(x) d_i(t)
\end{equation}

Consider the subspaces $D^h \subset D$ and $V^h \subset V$ of linear
interpolations defined above and time-invariant interpolations respectively.
It can be shown that $u^d(x, t)$ is a solution of the weak formulation over the
sets $D^h$ and $V^h$, where $d$ evolves
according to the following linear system:

\begin{equation}\label{eq:fem}
    \Sigma^H_{FEM}(u_0, g) \quad \left \{
    \begin{aligned}
        \dot{d} &= A d + b(g) \\
        d_i(0) &= u_0(x_i), i = 1,...,n
    \end{aligned}
    \right.
\end{equation}

In the above, $A = -M^{-1}K, b(g) = M^{-1} F(g)$ and $M, K$ and $F(g)$ are the capacity,
stiffness and external force matrices respectively. For the sake of brevity, 
we omit the definition of these matrices. However, we want to point out that,
in this case, $A$ is a banded matrix of bandwidth 3. Other PDEs share a similar
coupled structure, where the evolution of the field value at each node only
depends on the values at adjacent nodes. Also note that the external force
matrix depends on the boundary conditions $g$, which also holds for other types
of (possibly time-varying) boundary conditions and other PDEs. Under the
linearity assumption on the PDE, the resulting ODE system is linear in the
boundary conditions, which we can view as control inputs.

\subsection{Abstraction-Based Approach}
\label{sub:abstraction_based_approach}

In this approach, we are given a PDE system, $\Sigma$, and a specification in a logic based on
LTL, $\psi$. We will define a partition of the domain of the system that is
consistent with the predicates in $\psi$ (i.e., the boundaries of the spatial
domains of the predicates coincide with boundaries of elements in the
partition), and obtain the FEM system associated with the partition,
$\Sigma_{FEM}$. This is a regular linear ODE system that can, in principle, be
analyzed using the classical formal methods theory. In short, we consider a partition of the
state space of $\Sigma_{FEM}$ which is used to abstract the dynamics of the
system into a finite-state model, the Transition System (TS) $T$. Then, we obtain the
product of $T$ with an automata representation of the specification, usually a
Büchi automaton $B$, resulting in the automaton $P = T \times B$. The product $P$ contains all
satisfying runs of the system when projected over $T$, and the verification or
control problem is reduced to a model checking or synthesis problem on the automaton $P$.

In order to fully convert the problem from a PDE system to the ODE system, we need a
corresponding regular LTL formula, $\psi_{FEM}$, with predicates now denoting regions of the
state space of $\Sigma_{FEM}$. In general, we will address this translation step
by considering the predicates to apply only to the field values of the PDE at
the nodes. For example, the predicate "the temperature of the plate at every
point in region
$X$ is less than \SI{100}{\kelvin}" would be translated to "the temperature at
all nodes within $X$ is less than \SI{100}{\kelvin}", which defines a region in
the state space of $\Sigma_{FEM}$ and so fits within the classic formal methods
framework for linear systems and LTL specifications.

\subsubsection{Conservative Language Translation}
\label{ssub:Language translation}

Note that in reducing the verification or control problem from a PDE system with
an extended LTL specification to a linear ODE system with a regular LTL
specification, we introduce an approximation error from the FEM and incur a loss of
information from the translation of the specification. As an example of 
the importance of the latter,
assume we synthesize a control scheme for the boundary temperatures of a steel plate
that satisfies the predicate above for $\Sigma_{FEM}$ by forcing the temperature
to be \SI{99}{\kelvin} at all nodes. It is possible, however, that due to the
dynamics of the PDE system, the temperature between nodes follows a wave pattern
that results in the plate reaching temperatures higher than \SI{100}{\kelvin} at
some points. 

In order to ensure the specification is satisfied by the PDE system, we will
design a translation procedure that takes into account the errors introduced by
the FEM approximation and the loss of information incurred by the translation
itself. The main idea will be to perturb the reference profiles of the
predicates to account for the worst case scenario. For example, if we can
guarantee that \SI{2}{\kelvin} is the maximum discrepancy between the real
temperature at a point in the plate and the approximated temperature at the
nodes, we would translate the predicate to "the temperature at all nodes within
$X$ is less than \SI{98}{\kelvin}". Note that the problem of how to perturb the
predicates in order to guarantee the satisfaction of the PDE system is not
trivial. For example, if the predicate was under a negation operator in the
specification, we would need to perturb it in the opposite direction, i.e., the new
reference temperature would be \SI{102}{\kelvin}. The reference profile itself
must also be taken into account if it is not flat.

The translation procedure introduces a level of conservatism that is similar to
the conservatism present in the classical abstraction-based verification and
control algorithms due to the quality of the partition of the state space. In
the case of PDE systems, this conservatism appears even before we apply the
abstraction-based algorithms, which leads to the problem of designing a ``good"
partition of the domain of the PDE. We will investigate how the parameters of
the PDE and the specification influence the conservatism of the translation
procedure and we will
draw upon the FEM literature to design a partitioning scheme that will reduce
the conservatism to a user-defined level in terms of distance to the reference
profiles.

\subsubsection{Transition System Decomposition}
\label{ssub:Transition System Decomposition}

As we mentioned before, one of the main challenges in this approach is the high
number of dimensions of the ODE system obtained from the FEM (which can range
from hundreds to tens of thousands in very accurate simulations). We propose a
decomposition method that avoids the expensive computation of the full
TS. Assume $\Sigma_{FEM}$ is the form of \cref{eq:fem}. We partition the domain
of the PDE in two connected sets, $L$ and $R$, which induce a partition of the state 
variables of $\Sigma_{FEM}$ in two corresponding sets, $d_L$
and $d_R$. We define the subsystems corresponding to each group as ODE systems
of the form:

\begin{equation}
    \Sigma^{L}_{FEM} : \dot{d}_L = A_L d_L + A_R d_R + b_L
\end{equation}

and similarly for the $R$ subsystem. In the above, $A_L$ and $A_R$ are matrices 
obtained from $A$
and $b_L$ is the vector obtained from the entries of $b$ corresponding to $x_L$.
Recall from \cref{sub:Finite Element Method} that $\Sigma_{FEM}$ has a special
coupled structure, which implies that in the previous equation, only the rows of $A_R$
corresponding to nodes near the boundary between $L$ and $R$ are not zero. Thus,
we reduce the original ODE system to two (or more, if we further refine the
partition of the domain) smaller ODE systems with some external variables.
Assuming the external variables are constrained to a finite set (which we can
compute by performing a reachability analysis), we abstract
the subsystems into the transition systems $T_L$ and $T_R$.

The problem is now be decomposed in two subproblems as long as we also
decompose the specification $\psi_{FEM}$ in two subspecifications,
$\psi_{FEM}^{L}$ and $\psi_{FEM}^{R}$, such that solving the subproblems yield a
solution to the full problem. A naive approach would be to remove all
predicates over variables in $d_R$ from $\psi_{FEM}$ to obtain $\psi_{FEM}^{L}$
and viceversa. We will investigate under which conditions this approach
allows us to lift the solutions of the subproblems to a solution of the full
problem. We will also explore alternative specification decompositions.
The lifting of subproblem solutions can also in itself be challenging, specially in
control problems. In some instances, the parallel composition of the control
strategies will be enough, but it is not clear whether this strategy will be
even feasible in more complicated examples.

\begin{example}
    Consider the evolution of the temperature in a rod of length \SI{5}{\cm} 
    partitioned every \SI{}{\cm} for the
    purposes of the FEM, and let $L$ and $R$ be its left and right halves
    respectively. The $L$ subsystem evolves according to the following equation:

    \begin{equation}
        \dot{d}_{1,2} = A_{1,2} d_{1,2} + A_3 d_3 + b_{1,2}
    \end{equation}

    and similarly for the $R$ subsystem. Assume that after we perform a reachability
    analysis on the system, we find $d_2$ and $d_3$ to be always within the set 
    $[10, 20]$. With this information, we can construct a TS for each subsystem
    based on a given partition of the state space. Suppose we want to verify the
    specification ``At all points in the rod, the temperature is always less
    than \SI{40}{\kelvin}", which is translated to the regular LTL formula
    $\psi_{FEM} = \Always \bigwedge_{i=1}^{4} d_i < 40$. We can verify each subsystem with
    its corresponding subspecification, which in the case of the $L$ subsystem
    is $\psi^L_{FEM} = \Always \bigwedge_{i=1}^{2} d_i < 40$. In this case, if
    each subsystem satisfies its subspecification, we can conclude the full
    system satisfies the specification.
\end{example}

\subsection{Optimization-Based Approach}
\label{sub:optimization_based_approach}

Our second approach to formal methods for PDEs will be based on optimization
techniques. Consider that we are given a specification in a spatial extension of
STL. We follow a procedure similar to \cref{sub:abstraction_based_approach} and
obtain an ODE system, $\Sigma_{FEM}$, that approximates the PDE field at the nodes, and a
translated STL formula with predicates over the state space of $\Sigma_{FEM}$.
In order to encode the verification or control problem as an optimization
problem, we further approximate the system with a set of difference equations
in order to obtain a time discretization of the system for some time interval
$\Delta t$. A theoretically
perfect discretization for a system in the form of \cref{eq:fem} is:

\begin{equation}
    \label{eq:disc_system}
    \Sigma_{FEM}^{\Delta t}(d(0), g) : \quad \left\{
    \begin{aligned}
        \tilde d^{k+1} &= \tilde A \tilde d^k + \tilde b(g) \\
        \tilde d^0 &= d(0)
    \end{aligned}
    \right.
\end{equation}

where $\tilde A = e^{A \Delta t}$ and $\tilde b = - e^{A \Delta t} A^{-1} \left
( e^{- A \Delta t} - I \right ) b$. However, in practice one needs to numerically compute the
exponential matrices in $\tilde A$ and $\tilde b$, which introduces an
approximation error difficult to control. As an alternative, we can use any
numerical integration algorithm with fixed time step appropriate to the specific
PDE system under study, several of which have been
thoroughly analyzed in the FEM literature, such as the Newmark family. However,
we require the difference equations to be linear.

The specification $\psi_{FEM}$ must also be translated to a discrete-time
specification, $\psi_{FEM}^{\Delta t}$, which will also be perturbed 
to compensate the approximation error as well as the information loss. Finally,
we can encode the evolution of the system $\Sigma_{FEM}^{\Delta t}$, as well as
the STL formula $\psi_{FEM}^{\Delta t}$, as mixed integer linear constraints,
following the procedure proposed in \cite{sadra's paper}. Recall that we can
define quantitative semantics for STL using the notion of robustness of a
specification $\phi$ with respect to a signal $\tilde d$, $r(\phi, \tilde d)$,
which is positive if and only if $\tilde d$ satisfies $\phi$. Then, a verification problem
for a set of initial conditions $D_0$ can be formulated as the following
optimization problem:

\begin{equation}
    \label{eq:stl_set_opt}
    \begin{aligned}
        &r_m = &\min \quad &r(\psi^{\Delta t}_{FEM}, \tilde{d}) \\
        &  &\text{s.t. } &\cref{eq:disc_system} \\
        &  & &d(0) \in D_0
    \end{aligned}
\end{equation}

where the system satisfies the specification for all initial conditions in $D_0$
if $r_m > 0$. A similar formulation will allow us to verify sets of static
boundary conditions instead, which is a problem that naturally arises in
mechanical systems where bodies are assumed to be initially at rest. 
If we are instead solving a control synthesis problem where we
consider the control inputs to be the boundary conditions, which we require to
be within a set $G$, we reformulate it as:

\begin{equation}
    \label{eq:stl_ctrl_opt}
    \begin{aligned}
        &r_M = &\max \quad &r(\psi^{\Delta t}_{FEM}, \tilde{d}) \\
        &  &\text{s.t. } &\cref{eq:disc_system} \\
        &  & &g^k \in G
    \end{aligned}
\end{equation}

where the synthesized controls satisfy the specification if $r_M > 0$.
Similarly, an optimal control problem with input step-cost $c$ can be formulated
as follows:

\begin{equation}
    \label{eq:stl_opt_ctrl_opt}
    \begin{aligned}
        &  &\min \quad &\sum_i c(g^i) \\
        &  &\text{s.t. } &\cref{eq:disc_system} \\
        &  & &r(\psi^{\Delta t}_{FEM}, \tilde{d}) > 0\\
        &  & &g^k \in G
    \end{aligned}
\end{equation}

Note that the final time for the optimization problems is given by the temporal
horizon of the specification, i.e., the maximum time required to decide over the
satisfaction of the specification. If the horizon is too long, a receding
horizon scheme can be employed, with the usual challenges of convergence to the
optimal solution, terminal constraints and recursive feasibility.

Some of the general challenges that we will need to address with this approach will be
the relationship between the optimal cost for the discretized system and the PDE
system, as well as the computational complexity to solve the optimization
problem. Regarding the latter, an important observation is the partition of the
PDE domain does not only influence the complexity directly through the resulting
dimension of the FEM system, but also through the choice of $\Delta t$, which is
usually required to be smaller as the size of the partition decreases. We will
consider this observation as part of the design of the partition, analyzing the
tradeoff between complexity and conservatism.

\end{document}
