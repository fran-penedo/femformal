\documentclass{article}
\usepackage{color}
\usepackage{graphicx}
\usepackage{amsmath}
\usepackage{amsfonts}
\usepackage{mathtools}
\usepackage[utf8]{inputenc}
\usepackage{algpseudocode}
\usepackage{algorithm}

% Environments
\newtheorem{theorem}{Theorem}
\newtheorem{definition}{Definition}
\newtheorem{problem}{Problem}
\newtheorem{remark}{Remark}

% Aliases

\newcommand*{\xd}{\dot{x}}
\newcommand*{\R}{\mathbb{R}}
\newcommand*{\N}{\mathbb{N}}

\newcommand*{\psat}[1]{[[#1]]}

\newcommand*{\fran}[1]{\textcolor{blue}{#1}}

\title{Placeholder}

\author{Francisco Penedo}

\begin{document}

Let $\Omega = (0, L) \subset \R$ be an open interval representing the interior
of a one-dimensional rod of length $L$; $\rho, c, \kappa > 0$ 
constants denoting density, capacity and conductivity of the rod's material respectively;
$g_0, g_L \in \R$ the boundary conditions at each end of the rod; and $u_0 :
\Omega \rightarrow \R$ an initial value for the temperature on the rod. 
The evolution of the temperature at
each point in the rod can be described by a function $u : \bar \Omega \times [0,
T] \rightarrow \R$, where $T > 0$ denotes the final time and can be infinity, 
such that the following equations are satisfied:

\begin{equation}\label{eq:pde}
    \begin{aligned}
        \rho c \frac{\partial u}{\partial t} - \kappa \frac{\partial^2
        u}{\partial^2 x} &= 0, \text{on } \Omega \times (0, T) \\
        u(0, t) &= g_0, \forall t \in (0, T) \\
        u(L, t) &= g_L, \forall t \in (0, T) \\
        u(x, 0) &= u_0(x), \forall x \in \Omega
    \end{aligned}
\end{equation}

In the following we will call $\Sigma$ a system governed by \eqref{eq:pde},
described by the parameters $L, \rho, c$ and $\kappa$, and with boundary
conditions $g_0, g_L$.

Let $\Pi = \{\pi_i | i = 1, ..., m\}$ be a set of propositions, where each
proposition $\pi_i$ is represented as a tuple $(A_i, p_i)$, with $A_i \subseteq
\Omega$ a closed set (of $\Omega$ with the subset topology) and $p_i : A_i
\rightarrow \R$ a continuous function, and denotes the following subset of
continuous functions supported in $\bar\Omega$, $C^0(\bar\Omega)$:

\begin{equation}
    \psat{\pi_i} = \{ f \in C^0(\bar\Omega) | \left.f\right|_{A_i} \sim_i p_i \},
\end{equation}

where $\left.f\right|_{A_i}$ denotes the restriction of $f$ to $A_i$ and $\sim_i
\in \{<, >\}$.

Consider an LTL formula over $\Pi$, $\phi$, with the usual definition of
propositional LTL over a set of atomic propositions and the usual semantics
given by the satisfaction of $\phi$ at position $i \in \N$ of word $w$, denoted
by $w(i) \models \phi$. We give a definition for the satisfaction of $\phi$ by a
continuous function $a : \bar \Omega \times [0, \infty) \rightarrow
\R$\footnote{\fran{This is mostly copied from Marius work with minor
adaptations}}. For each $\Theta \in 2^\Pi$, let $\psat{\Theta}$ be the set of
continuous functions supported in $\bar\Omega$ satisfying all and only
propositions $\pi \in \Theta$:

\begin{equation}
    \psat{\Theta} = \bigcap_{\pi \in \Theta} \psat{\pi} \setminus 
    \bigcup_{\pi \in \Pi \setminus \Theta} \psat{\pi}
\end{equation}

In the following we use the notation $a(t)$, with $t \in [0, \infty)$ to denote
the function $a(\cdot , t) : \bar\Omega \rightarrow \R$.

\begin{definition}\label{def:word}
    The word corresponding to function $a$ is the sequence $w_a = w_a(1)
    w_a(2)..., w_a(k) \in 2^\Pi, k \geq 1$, generated according to the following
    rules, which must be satisfied for all $\tau \geq 0$ and $k \geq 1$:

    \begin{itemize}
        \item $a(0) \in \psat{w_a(1)}$;
        \item if $a(\tau) \in \psat{w_a(k)}$ and $w_a(k) \neq w_a(k + 1)$, then
            there exists $\tau' > \tau$ such that: a) $a(\tau') \in \psat{w_a(k
            + 1)}$, b) $a(t) \notin \psat{\pi}, \forall t \in [\tau, \tau'], \forall \pi \in
            \Pi \setminus (w_a(k) \cup w_a(k + 1))$, and c) \fran{[not sure how
            to adapt this one; mostly technical]};
        \item if $a(t) \in \psat{w_a(k)}$ and $w_a(k) = w_a(k + 1)$, then $a(t) \in
        \psat{w_a(k)}, \forall t \geq \tau$.
    \end{itemize}
\end{definition}

\begin{definition}\label{def:sat}
    A function $a : \bar \Omega \times [0, \infty) \rightarrow \R$ satisfies an
        LTL formula $\phi$, denoted as $a \models \phi$, if and only if $w_a
        \models \phi$, where $w_a$ is the word corresponding to $a$.
\end{definition}

\begin{problem}\label{pr:pde}
    Given a system $\Sigma$ as in \eqref{eq:pde}, an LTL formula $\phi$ and an
    initial region represented as $\Theta \in 2^\Pi$, check whether the trajectories of $\Sigma$
    satisfy $\phi$, for all $u_0 \in \psat{\Theta}$.
\end{problem}

We now make use of the Finite Element Analysis framework to reformulate
Problem~\ref{pr:pde} in terms of a system of ODEs. Let $\{x_i\}_{i = 0}^{n +
1}$, where $x_0 = 0, x_{n+1} = L, x_i \in \Omega, i = 1,...,n$, be a partition of
$\bar\Omega$ such that it is proposition preserving with respect to the set of
regions $\{A_i\}_{i = 1}^{m}$\footnote{This is well defined since the $A_i$ are
closed and there are a finite number of them}. Let $d_i(t), i = 1,...,n$ represent the
temperature of the rod at the point $x_i$, and let $d = (d_1, ..., d_n)' \in
\R^n$. It can be shown\footnote{\fran{At some point I'll fill in the details of the
construction of $A$ and $b$}} that $d_i(t)$ is a good approximation of 
$u(x, t)$ at $x=x_i$, where $u(x,t)$ is the trajectory of $\Sigma$ for an
initial value $u_0$, and $d$ evolves
according to the following linear system, $\Sigma_{FEM}$:

\begin{equation}\label{eq:fem}
    \begin{aligned}
        \dot{d} &= A d + b
    \end{aligned}
\end{equation}

with initial value $d_i(0) = u_0(x_i), i = 1,...,n$. In the above, $A$ is a
banded matrix of bandwidth 3. We call $\Sigma_{FEM}$ the FEM system
corresponding to system $\Sigma$.

In order to fully reformulate Problem~\ref{pr:pde}, we need to modify $\phi$ so
that it can be checked against trajectories of \eqref{eq:fem}. Let $\Pi_{FEM} =
\{\pi_i^j | i =1,...,m, j \in J_i\}$, where $J_i = \{j | x_j \in A_i\}$ and
$\pi_i^j$ denotes the following region in $\R^n$:

\begin{equation}
    \psat{\pi_i^j} = \{d \in \R^n | d_j \sim_i p_i(x_j)\}
\end{equation}

\begin{definition}\label{def:femformula}
    The LTL formula over $\Pi_{FEM}$, $\phi_{FEM}$, corresponding to an LTL formula
    $\phi$ over $\Pi$, is a formula obtained by substituting every atomic
    proposition $\pi_i$ in $\phi$ by the formula $\bigwedge_{j \in J_i}
    \pi_i^j$.
\end{definition}

We define the word corresponding to a trajectory of \eqref{eq:fem} and
satisfaction of a formula $\phi_{FEM}$ by a trajectory in a similar way as in
Def.~\ref{def:word} and \ref{def:sat}. An immediate consequence of
Def.~\ref{def:femformula} is the following:

\begin{theorem}\label{th:equiv}
    Let $a : \bar\Omega \times [0, \infty) \rightarrow \R$ be a function
    satisfying the LTL formula over $\Pi$, $\phi$. Then, the trajectory $d : [0,
    \infty) \rightarrow \R^n$ given by $d_i(t) = a(x_i, t), t \geq 0, i =
    1,...,n$, satisfies the corresponding LTL formula over $\Pi_{FEM}$, $\phi_{FEM}$.
\end{theorem}

We can now formulate a corresponding problem to Problem~\ref{pr:pde} in terms of 
the corresponding FEM system and LTL formula over $\Pi_{FEM}$:

\begin{problem}\label{pr:fem}
    Given a system $\Sigma$, an LTL formula $\phi$ over $\Pi$ and an initial
    region $\Theta \in 2^\Pi$, check whether the trajectories of the corresponding
    FEM system, $\Sigma_{FEM}$, satisfies the corresponding LTL formula over
    $\Pi_{FEM}$, $\phi_{FEM}$, for all initial values corresponding to the
    initial values $u_0 \in \psat{\Theta}$ of $\Sigma$.
\end{problem}

\begin{theorem}
    If the solution to Problem~\ref{pr:fem} is SAT, then the solution to
    Problem~\ref{pr:pde} is SAT.
\end{theorem}

\fran{As of now, this theorem is clearly false: we need to account for the
quality of the FEM approximation. I think this can be done if 
we can compute error bounds for $\|d_i - u(x_i, \cdot)\|$ and we adjust the
propositions in $\Pi_{FEM}$ so that the converse of Th.~\ref{th:equiv} is true.}

\end{document}
