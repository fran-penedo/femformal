
\iffalse
\section{Approach}

In the following, we assume we are given a system $\Sigma$ and an LTL formula
$\phi$ over a suitable set $\Pi$, from which we have obtained the 
FEM system $\Sigma_{FEM}$ and LTL formula $\phi_{FEM}$ over $\Pi_{FEM}$ for a
given partition. We also assume there is a hyperrectangle $R = \{d \in \R^n |
d_i \in (r_{min}, r_{max}), i = 1,...,n\}$ that is invariant for $\Sigma_{FEM}$.

\fran{The above may not actually be needed at all}

Let $T = \{ p_i(x_j) | i = 1,...,m, j \in J_i \} \cup \{r_{min}, r_{max}\} =
\{t_i | i = 1,..., |T|\}$, with $t_i < t_{i+1}$, represent a partition of $[r_{min},
r_{max}]$, which can be extended to a partition of $R$ into hyperrectangles. We
denote each hyperrectangle in the following way:

\begin{equation}
    R_{(a_j)_{j=1}^n} = \{d \in \R^n | d_i \in (t_{a_i}, t_{a_i + 1})\},
\end{equation}

where $a_j \in \{1,...,n\}, j=1,...,n$. We can now define the quotient
transition system associated with $\Sigma_{FEM}$ and partition $T$,
$TS = (Q, Q_0, \rightarrow, O, o)$, where:

\begin{itemize}
    \item $Q = \{(a_j)_{j=1}^n | a_j \in \{1,...,|T|\}\}$ is the set of states.
    \item $Q_0 \subseteq Q$ is the set of initial states.
    \item $(q,q') \in \rightarrow$ if and only if there exists $d \in R_q$ such
        that the trajectory of $\Sigma_{FEM}$ with initial value $d$ leaves
        $R_q$ in finite time and enters $R_{q'}$ immediately
        after\footnote{\fran{This is not formal enough}}.
    \item $O = 2^{\Pi_{FEM}}$ is the set of observations.
    \item $o(q) = \Theta$ if and only if $R_q \subseteq [[\Theta]]$.
\end{itemize}

We define the following set of equivalence relations in $Q$:

\begin{equation}
    ((a_j)_{j=1}^n,(b_j)_{j=1}^n) \in S_I \iff a_j = b_j, \forall j \in I
\end{equation}

And the corresponding quotient transition systems $TS/S_I = (Q/S_I, Q_0/S_I,
\to/S_I, O/S_I, o/S_I)$. Here, $O/S_I$ is the power set of the set of atomic
propositions related to state variable with index in $I$, i.e., the set
$\{\pi_i^j \in \Pi_{FEM} | j \in I \}$. This ensures that $o/S_I$ has a unique
value for each equivalence class in $Q/S_I$.

Let $\psi$ be an LTL formula over $\Pi_{FEM}$. We
denote by $\Gamma(\phi_{ij})$ the set of coordinate indices that appear in
atomic propositions in $\phi_{ij}$, i.e.,

\begin{equation}
    \Gamma(\psi) = \{j \in \{1,...,n\}| \text{ some }\pi_i^j \in \Pi_{FEM} \text{ appears in } \psi\}
\end{equation}


\begin{theorem}

Let $\phi = \bigvee \bigwedge \phi_{ij}$ be an LTL formula over $\Pi_{FEM}$. If
there exists $i$ such that for all $j$, some overapproximation of the quotient transition system
$TS/S_{\Gamma(\phi_{ij})}$ satisfies $\phi_{ij}$, then the transition system $TS$
satisfies $\phi$.
    
\end{theorem}

\begin{proof}

Suppose by contradiction that $TS$ does not satisfy $\phi$, i.e., there exists a
trajectory in $TS$, $\mathbf{q} = q_0 q_1...$ such that the corresponding observed
word $\mathbf{o} = o(q_0) o(q_1)...$ does not satisfy $\phi$. Due to the structure
of $\phi$, this implies that for all $i$ there exists $j$ such that $\mathbf{o}$ does
not satisfy $\phi_{ij}$. Let $\mathbf{o^{ij}}$ be the word obtained from
$\mathbf{o}$ by substituting every symbol not in $O/S_{\Gamma{\phi_{ij}}}$ by
the last preceding symbol that was in the set. Then, $\mathbf{o^{ij}}$ is a word
produced by $TS/S_{\Gamma(\phi_{ij})}$, which implies the latter transition system
does not satisfy $\phi_{ij}$. This contradicts the hypothesis of the
theorem.
    
\end{proof}

The previous theorem allows us to avoid the computation of the full transition
system $TS$ when checking the satisfaction of a formula structured as a boolean
combination of LTL formulas over atomic propositions referring to a small subset
of the state variables. Instead, we only need overapproximations of the quotient
transition systems, which can be computed without $TS$, as explained below. Also
note that we can use any valid rule of inference in LTL in order to rewrite the
formula so that the components $\phi_{ij}$ have a smaller $\Gamma$ set.

Let $I \subseteq \{1,...,n\}$ be a set of indices. 

\begin{definition}

The subsystem of $\Sigma_{FEM}$ corresponding to $I$,
$\Sigma_{FEM}^I$, is a linear system that evolves according to the following
equation:

\begin{equation}
    \dot{d_I} = A_I d_I + b_I + \tilde{A_I}\tilde{d_I},
\end{equation}

where $d_I \in \R^{|I|}$, $A_I = (a_{ij})_{i, j \in I}$, $b_I = (b_i)_{i
\in I}$, $\tilde{d_I} \in \R^{|\tilde{I}|} $, $\tilde{A_I} = (a_{ij})_{i \in I, j
\in \tilde{I_i}}$, $\tilde{I_i} = \{j \in \{1,...,n\} \setminus I | a_{ij} \neq 0\}$, for $i
\in I$ and $\tilde{I} = \bigcup_{i \in I} \tilde{I_i}$\footnote{\fran{All the
$\tilde{I_i}$ mess is sort of unnecessary for this definition, since using all
indices not in $I$ yields an equivalent system (everything extra is multiplied
by 0). However, this definition makes explicit the fact that a subsystem
corresponding to $I$ only cares about its own dimensions and the dimensions it's
related with via $A$.}}.

\end{definition}

% \begin{definition}
%
% The projection of $\phi_{FEM}$ onto $I$ is an LTL
% formula over $\Pi_{FEM}^I = \{\pi_i^j | i = 1,...,m, j \in J_i \cap I\}$,
% $\phi_{FEM}^I$, obtained by removing every atomic proposition not in
% $\Pi_{FEM}^I$ from $\phi_{FEM}$.
%
% \end{definition}
%
% \begin{problem}
%
% Solve Problem~\ref{pr:fem} for subsystem $\Sigma_{FEM}^I$ and formula
% $\phi_{FEM}^I$, considering $\tilde{d_I}$ a disturbance whose components are
% all contained in $(r_{min}, r_{max})$. \fran{Review this to state what
% a disturbance is.}
%     
% \end{problem}
%

\begin{theorem}

The quotient transition system associated with $\Sigma_{FEM}^I$ and
partition $T$, $TS^I$, is an overapproximation of $TS/S_I$.
    
\end{theorem}

\fi

\iffalse
\section{Examples [Outdated]}
% \label{sec:examples}

Consider a rod with parameters given in Table~\ref{tab:ex_pars}. We define
the following atomic propositions: 
\begin{equation}
    \begin{aligned}
        \pi_1 &= ([0, 500], <, 85) \\
        \pi_2 &= ([500, 1000], <, 125) \\
        \pi_3 &= ([0, 1000], <, 25) \\
    \end{aligned}
\end{equation}

We want to check the satisfaction of the following specifications for the
initial region $\{\pi_3\}$:

\begin{equation}
\label{eq:ex_specs}
    \begin{aligned}
        \phi &= \Always \pi_1 \wedge \Always \pi_2 \\ 
    \end{aligned}
\end{equation}

We use as our FEM partition the set $\{k | k = 0,...,1000\}$, which results
in a FEM system with the following matrices:

\begin{equation}
    A = - \frac{6} {5} \begin{pmatrix}
        2 & -1  & 0 & \cdots \\ 
        -1 & 2 & -1 & \ddots \\ 
        0 & -1 & 2 & \ddots \\
        \vdots & \ddots & \ddots & \ddots 
    \end{pmatrix},
    b = \frac{6} {5}\begin{pmatrix}
        10 \\
        0 \\
        \vdots \\
        0 \\
        100
    \end{pmatrix}
\end{equation}

The corresponding FEM formulas for the specifications given in
\eqref{eq:ex_specs} are:

\begin{equation}
    \begin{aligned}
        \phi_{FEM} = (\Always \bigwedge_{i = 1}^{500} d_i < 85) \wedge (\Always
            \bigwedge_{i = 500}^{999} d_i < 125)
    \end{aligned}
\end{equation}

Note that $\phi_{FEM}$ has the form $\phi_{FEM} = \phi^1_{FEM} \wedge
\phi^2_{FEM}$, with $\Gamma(\phi^1_{FEM}) = \{1,...,500\}, \Gamma(\phi^1_{FEM})
= \{500,...,999\}$. These $\Gamma$ sets are too big to compute the corresponding
quotients in Th~\ref{th:partition}. However, we can manipulate $\phi_{FEM}$ to obtain the
following formula:

\begin{equation}
    \phi_{FEM} = (\bigwedge_{i = 1}^{500} \Always (d_i < 85)) \wedge
    (\bigwedge_{i = 500}^{999} \Always (d_i < 125))
\end{equation}

which has the form $\phi_{FEM} = \bigwedge_{i=1}^{1000} \bar{\phi}^i_{FEM}$ and
each subformula $\bar{\phi}^i_{FEM}$ has a $\Gamma$ set of size 1. The problem
is now reduced to check the satisfaction of $\bar{\phi}^i_{FEM}$ in the TS
$TS/S_{\Gamma(\bar{\phi}^i{FEM})}$, for all $i=1,...,1000$. We use $T =
\{10, 15, 25,...,105,110\}$ for the temperature partition. We show in
Fig~\ref{fig:ex_ts} some of the resulting transition systems. The model-checking
tool returned UNSAT for some instances, so we cannot conclude that $\phi_{FEM}$ is
satisfied by $\Sigma_{FEM}$. To see why this is the case, notice the transition
system in Fig~\ref{fig:ex_ts} contains all posible transitions, i.e., it does
not give us any information about the subsystem it abstracts. Consequently, the
corresponding formula $\phi_{FEM}^{xxxx} = \Always (d_{500} < 85)$ is not
satisfied.
\fi

\iffalse

In order to fully reformulate Problem~\ref{pr:pde}, we need to modify $\phi$ so
that it can be checked against trajectories of \eqref{eq:fem}. Let $\Pi_{FEM} =
\{\pi_i^j | i =1,...,m, j \in J_i\}$, where $J_i = \{j | x_j \in X_i\}$ and
$\pi_i^j$ denotes the following region in $\R^n$:

\begin{equation}
    \psat{\pi_i^j} = \{d \in \R^n | d_j \sim_i p_i(x_j)\}
\end{equation}

\begin{definition}\label{def:femformula}
    The LTL formula over $\Pi_{FEM}$, $\phi_{FEM}$, corresponding to an LTL formula
    $\phi$ over $\Pi$, is a formula obtained by substituting every atomic
    proposition $\pi_i$ in $\phi$ by the formula $\bigwedge_{j \in J_i}
    \pi_i^j$.
\end{definition}

We define the word corresponding to a trajectory of \eqref{eq:fem} and
satisfaction of a formula $\phi_{FEM}$ by a trajectory in a similar way as in
Def.~\ref{def:word} and \ref{def:sat}. An immediate consequence of
Def.~\ref{def:femformula} is the following:

\begin{theorem}\label{th:equiv}
    Let $a : \bar\Omega \times [0, \infty) \rightarrow \R$ be a function
    satisfying the LTL formula over $\Pi$, $\phi$. Then, the trajectory $d : [0,
    \infty) \rightarrow \R^n$ given by $d_i(t) = a(x_i, t), t \geq 0, i =
    1,...,n$, satisfies the corresponding LTL formula over $\Pi_{FEM}$, $\phi_{FEM}$.
\end{theorem}

We can now formulate a corresponding problem to Problem~\ref{pr:pde} in terms of 
the corresponding FEM system and LTL formula over $\Pi_{FEM}$:

\begin{problem}\label{pr:fem}
    Given a system $\Sigma$, an LTL formula $\phi$ over $\Pi$ and an initial
    region $\Theta \in 2^\Pi$, check whether the trajectories of the corresponding
    FEM system, $\Sigma_{FEM}$, satisfies the corresponding LTL formula over
    $\Pi_{FEM}$, $\phi_{FEM}$, for all initial values corresponding to the
    initial values $u_0 \in \psat{\Theta}$ of $\Sigma$.
\end{problem}

\begin{theorem}
    If the solution to Problem~\ref{pr:fem} is SAT, then the solution to
    Problem~\ref{pr:pde} is SAT.
\end{theorem}

\fran{As of now, this theorem is clearly false: we need to account for the
quality of the FEM approximation. I think this can be done if 
we can compute error bounds for $\|d_i - u(x_i, \cdot)\|$ and we adjust the
propositions in $\Pi_{FEM}$ so that the converse of Th.~\ref{th:equiv} is true.}

\fi
\iffalse

\subsection{Linear Temporal Logic}
\label{sub:linear_temporal_logic}

Let $\Pi = \{\pi_i | i = 1, ..., m\}$ be a set of predicates, where each
predicate $\pi_i$ is represented as a tuple $(X_i, \sim_i, p_i)$, with $X_i \subseteq
\Omega$ a closed set (of $\Omega$ with the subset topology) and $p_i : X_i
\rightarrow \R$ a continuous function, and denotes the following subset of
continuous functions supported in $\bar\Omega$, $C^0(\bar\Omega)$:

\begin{equation}
    \psat{\pi_i} = \{ f \in C^0(\bar\Omega) | \left.f\right|_{X_i} \sim_i p_i \},
\end{equation}

where $\left.f\right|_{X_i}$ denotes the restriction of $f$ to $X_i$ and $\sim_i
\in \{<, >\}$.

Consider an LTL formula over $\Pi$, $\phi$, with the usual definition of
propositional LTL over a set of atomic propositions and the usual semantics
given by the satisfaction of $\phi$ at position $i \in \N$ of word $w$, denoted
by $w(i) \models \phi$. We give a definition for the satisfaction of $\phi$ by a
continuous function $a : \bar \Omega \times [0, \infty) \rightarrow
\R$\footnote{\fran{This is mostly copied from Marius work with minor
adaptations}}. For each $\Theta \in 2^\Pi$, let $\psat{\Theta}$ be the set of
continuous functions supported in $\bar\Omega$ satisfying all and only
propositions $\pi \in \Theta$:

\begin{equation}
    \psat{\Theta} = \bigcap_{\pi \in \Theta} \psat{\pi} \setminus 
    \bigcup_{\pi \in \Pi \setminus \Theta} \psat{\pi}
\end{equation}

In the following we use the notation $a(t)$, with $t \in [0, \infty)$ to denote
the function $a(\cdot , t) : \bar\Omega \rightarrow \R$.

\begin{definition}\label{def:word}
    The word corresponding to function $a$ is the sequence $w_a = w_a(1)
    w_a(2)..., w_a(k) \in 2^\Pi, k \geq 1$, generated according to the following
    rules, which must be satisfied for all $\tau \geq 0$ and $k \geq 1$:

    \begin{itemize}
        \item $a(0) \in \psat{w_a(1)}$;
        \item if $a(\tau) \in \psat{w_a(k)}$ and $w_a(k) \neq w_a(k + 1)$, then
            there exists $\tau' > \tau$ such that: a) $a(\tau') \in \psat{w_a(k
            + 1)}$, b) $a(t) \notin \psat{\pi}, \forall t \in [\tau, \tau'], \forall \pi \in
            \Pi \setminus (w_a(k) \cup w_a(k + 1))$, and c) \fran{[not sure how
            to adapt this one; mostly technical]};
        \item if $a(t) \in \psat{w_a(k)}$ and $w_a(k) = w_a(k + 1)$, then $a(t) \in
        \psat{w_a(k)}, \forall t \geq \tau$.
    \end{itemize}
\end{definition}

\begin{definition}\label{def:sat}
    A function $a : \bar \Omega \times [0, \infty) \rightarrow \R$ satisfies an
        LTL formula $\phi$, denoted as $a \models \phi$, if and only if $w_a
        \models \phi$, where $w_a$ is the word corresponding to $a$.
\end{definition}

\begin{problem}\label{pr:pde}
    Given a system $\Sigma$ as in \eqref{eq:pde}, an LTL formula $\phi$ and an
    initial region represented as $\Theta \in 2^\Pi$, check whether the trajectories of $\Sigma$
    satisfy $\phi$, for all $u_0 \in \psat{\Theta}$.
\end{problem}

\fi

